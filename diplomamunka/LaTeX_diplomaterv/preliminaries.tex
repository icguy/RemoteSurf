%--------------------------------------------------------------------------------------
% Rovid formai es tartalmi tajekoztato
%--------------------------------------------------------------------------------------

\footnotesize
\begin{center}
\large
\textbf{\Large Általános információk, a diplomaterv szerkezete}\\
\end{center}

A diplomaterv szerkezete a BME Villamosmérnöki és Informatikai Karán:
\begin{enumerate}
\item	Diplomaterv feladatkiírás
\item	Címoldal
\item	Tartalomjegyzék
\item	A diplomatervezõ nyilatkozata az önálló munkáról és az elektronikus adatok kezelésérõl
\item	Tartalmi összefoglaló magyarul és angolul
\item	Bevezetés: a feladat értelmezése, a tervezés célja, a feladat indokoltsága, a diplomaterv felépítésének rövid összefoglalása
\item	A feladatkiírás pontosítása és részletes elemzése
\item	Elõzmények (irodalomkutatás, hasonló alkotások), az ezekbõl levonható következtetések
\item	A tervezés részletes leírása, a döntési lehetõségek értékelése és a választott megoldások indoklása
\item	A megtervezett mûszaki alkotás értékelése, kritikai elemzése, továbbfejlesztési lehetõségek
\item	Esetleges köszönetnyilvánítások
\item	Részletes és pontos irodalomjegyzék
\item	Függelék(ek)
\end{enumerate}

Felhasználható a következõ oldaltól kezdõdõ \LaTeX-Diplomaterv sablon dokumentum tartalma. 

A diplomaterv szabványos méretû A4-es lapokra kerüljön. Az oldalak tükörmargóval készüljenek (mindenhol 2.5cm, baloldalon 1cm-es kötéssel). Az alapértelmezett betûkészlet a 12 pontos Times New Roman, másfeles sorközzel.

Minden oldalon - az elsõ négy szerkezeti elem kivételével - szerepelnie kell az oldalszámnak.

A fejezeteket decimális beosztással kell ellátni. Az ábrákat a megfelelõ helyre be kell illeszteni, fejezetenként decimális számmal és kifejezõ címmel kell ellátni. A fejezeteket decimális aláosztással számozzuk, maximálisan 3 aláosztás mélységben (pl. 2.3.4.1.). Az ábrákat, táblázatokat és képleteket célszerû fejezetenként külön számozni (pl. 2.4. ábra, 4.2 táblázat vagy képletnél (3.2)). A fejezetcímeket igazítsuk balra, a normál szövegnél viszont használjunk sorkiegyenlítést. Az ábrákat, táblázatokat és a hozzájuk tartozó címet igazítsuk középre. A cím a jelölt rész alatt helyezkedjen el.

A képeket lehetõleg rajzoló programmal készítsék el, az egyenleteket egyenlet-szerkesztõ segítségével írják le (A \LaTeX~ehhez kézenfekvõ megoldásokat nyújt).

Az irodalomjegyzék szövegközi hivatkozása történhet a Harvard-rendszerben (a szerzõ és az évszám megadásával) vagy sorszámozva. A teljes lista névsor szerinti sorrendben a szöveg végén szerepeljen (sorszámozott irodalmi hivatkozások esetén hivatkozási sorrendben). A szakirodalmi források címeit azonban mindig az eredeti nyelven kell megadni, esetleg zárójelben a fordítással. A listában szereplõ valamennyi publikációra hivatkozni kell a szövegben (a \LaTeX-sablon a Bib\TeX~segítségével mindezt automatikusan kezeli). Minden publikáció a szerzõk után a következõ adatok szerepelnek: folyóirat cikkeknél a pontos cím, a folyóirat címe, évfolyam, szám, oldalszám tól-ig. A folyóirat címeket csak akkor rövidítsük, ha azok nagyon közismertek vagy nagyon hosszúak. Internet hivatkozások megadásakor fontos, hogy az elérési út elõtt megadjuk az oldal tulajdonosát és tartalmát (mivel a link egy idõ után akár elérhetetlenné is válhat), valamint az elérés idõpontját.

\vspace{5mm}
Fontos:
\begin{itemize}
	\item A szakdolgozat készítõ / diplomatervezõ nyilatkozata (a jelen sablonban szereplõ szövegtartalommal) kötelezõ elõírás Karunkon ennek hiányában a szakdolgozat/diplomaterv nem bírálható és nem védhetõ !
	\item Mind a dolgozat, mind a melléklet maximálisan 15 MB méretû lehet !
\end{itemize}

\vspace{5mm}
\begin{center}
Jó munkát, sikeres szakdolgozat készítést ill. diplomatervezést kívánunk !
\end{center}

\normalsize

%--------------------------------------------------------------------------------------
% Feladatkiiras (a tanszeken atveheto, kinyomtatott valtozat)
%--------------------------------------------------------------------------------------
\clearpage
\begin{center}
\large
\textbf{FELADATKIÍRÁS}\\
\end{center}

A feladatkiírást a tanszéki adminisztrációban lehet átvenni, és a leadott munkába eredeti, tanszéki pecséttel ellátott és a tanszékvezetõ által aláírt lapot kell belefûzni (ezen oldal \emph{helyett}, ez az oldal csak útmutatás). Az elektronikusan feltöltött dolgozatban már nem kell beleszerkeszteni ezt a feladatkiírást.

\begin{flushright}
 \vspace*{1cm}
 \makebox[7cm]{\rule{6cm}{.4pt}}\\
 \makebox[7cm]{\emph{dr. Tanszévezetõ Gábor}}\\
 \makebox[7cm]{tanszékvezetõ egyetemi tanár}
\end{flushright}
\vfill



